%%%%%%%%%%%%%%%%%%%%%%%%%%%%%%%%%%%%%%%%%
% Beamer Presentation
% LaTeX Template
% Version 1.0 (10/11/12)
%
% This template has been downloaded from:
% http://www.LaTeXTemplates.com
%
% License:
% CC BY-NC-SA 3.0 (http://creativecommons.org/licenses/by-nc-sa/3.0/)
%
%%%%%%%%%%%%%%%%%%%%%%%%%%%%%%%%%%%%%%%%%

%----------------------------------------------------------------------------------------
%	PACKAGES AND THEMES
%----------------------------------------------------------------------------------------

\documentclass{beamer}
%\documentclass[parskip]{scrartcl}
%\documentclass{article}
\mode<presentation> {

% The Beamer class comes with a number of default slide themes
% which change the colors and layouts of slides. Below this is a list
% of all the themes, uncomment each in turn to see what they look like.

%\usetheme{default}
%\usetheme{AnnArbor}
%\usetheme{Antibes}
%\usetheme{Bergen}
%\usetheme{Berkeley}
%\usetheme{Berlin}
%\usetheme{Boadilla}
%\usetheme{CambridgeUS}
%\usetheme{Copenhagen}
%\usetheme{Darmstadt}
%\usetheme{Dresden}
%\usetheme{Frankfurt}
%\usetheme{Goettingen}
%\usetheme{Hannover}
%\usetheme{Ilmenau}
%\usetheme{JuanLesPins}
%\usetheme{Luebeck}
\usetheme{Madrid}
%\usetheme{Malmoe}
%\usetheme{Marburg}
%\usetheme{Montpellier}
%\usetheme{PaloAlto}
%\usetheme{Pittsburgh}
%\usetheme{Rochester}
%\usetheme{Singapore}
%\usetheme{Szeged}
%\usetheme{Warsaw}
\usefonttheme{professionalfonts} 
%\usecolortheme{albatross}
\usecolortheme{beaver}
%\usecolortheme{beetle}
%\usecolortheme{crane}
%\usecolortheme{dolphin}
%\usecolortheme{dove}
%\usecolortheme{fly}
%\usecolortheme{lily}
%\usecolortheme{orchid}
%\usecolortheme{rose}
%\usecolortheme{seagull}
%\usecolortheme{seahorse}
%\usecolortheme{whale}
%\usecolortheme{wolverine}

%\setbeamertemplate{footline} % To remove the footer line in all slides uncomment this line
%\setbeamertemplate{footline}[page number] % To replace the footer line in all slides with a simple slide count uncomment this line

%\setbeamertemplate{navigation symbols}{} % To remove the navigation symbols from the bottom of all slides uncomment this line
}

\usepackage{graphicx} % Allows including images
\usepackage{booktabs} % Allows the use of \toprule, \midrule and \bottomrule in tables
\usepackage{amsmath}
\usepackage{amsfonts}
\usepackage{amssymb}
\usepackage{adjustbox}
\usepackage{mathtools}
\usepackage{caption}
\usepackage{multicol}
\usepackage{lipsum}
%\usepackage[T1]{fontenc}
\usepackage{eso-pic}
\usepackage{lmodern}
\usepackage{mathtools}
\usepackage{mdframed}
\usepackage{fancybox}
\usepackage{tikz}
%\usepackage[T1]{fontenc}
\usepackage[utf8]{inputenc}
\usepackage{etoolbox}
\usepackage{changepage}
\usepackage{array,colortbl,xcolor}
\usepackage{colortbl}
\usepackage{color}
\usepackage{verbatim} 
\usepackage{longtable}
\usepackage[english]{babel}
\usepackage{subfig}
\usepackage{cancel}
\usepackage[printwatermark]{xwatermark}
\usepackage{draftwatermark}
\usepackage{pst-abspos,pst-text}
\usepackage{transparent}
\usepackage{rotating}
\usepackage[printwatermark]{xwatermark}
\usepackage{draftwatermark}
\usepackage{tcolorbox}
\usepackage[customcolors]{hf-tikz}
\usetikzlibrary{arrows,shapes}
\SpecialCoor
%----------------------------------------------------------------------------------------
%	TITLE PAGE
%----------------------------------------------------------------------------------------

\title[PhD Defence]{Measurement of Top Quark Mass and Production Cross-Section using tt-bar pairs with Semi-Leptonic Decay Channel} % The short title appears at the bottom of every slide, the full title is only on the title page
%\begin{center}
%TOP-16-006
%\end{center}
\author[Qamar-ul-Hassan]{\\
\vspace{5 mm}
Qamar-ul-Hassan\\11-FBAS/PHDPHY/S11} % Your name
\institute[IIU-Islamabad] % Your institution as it will appear on the bottom of every slide, may be shorthand to save space
{
International Islamic University, Islamabad% \\ % Your institution for the title page
\medskip
%\textit{qamar@cern.ch} % Your email address
}
%\newcommand\AtPagemyUpperLeft[1]{\AtPageLowerLeft{%
%\put(\LenToUnit{0.01\paperwidth},\LenToUnit{0.9\paperheight}){#1}}}
%\AddToShipoutPictureFG{
%  \AtPagemyUpperLeft{{\includegraphics[width=.9cm,keepaspectratio]{paper_fig/CMScol}}}
%}%
%\titlegraphic{\includegraphics[width=2cm]{paper_fig/CMScol}%\hspace*{4.75cm}~%
%   \includegraphics[width=2cm]{paper_fig/CMScol}
%}
\institute [IIU-Islamabad]{Supervisor:{\bf{ Prof. Dr. Hafeez R. Hoorani}} (NCP, Islamabad)
\\Co-Supervisor:{\bf{ Dr. Waqar Adil Syed}} (IIU, Islamabad)}

\date{\today} % Date, can be changed to a custom date

\begin{document}

\begin{frame}
\titlepage % Print the title page as the first slide
\end{frame}
%----------------------------------------------------------------------------------------
%	PRESENTATION SLIDES
%----------------------------------------------------------------------------------------
\setbeamertemplate{itemize items}[triangle]
\newcommand{\tikzmark}[1]{\tikz[overlay,remember picture] \node (#1) {};}
\definecolor{mycolor}{RGB}{255,255,255}
\tcbset{colframe=white,colback=white,nobeforeafter}

\newcommand<>{\fullsizegraphic}[1]{
  \begin{textblock*}{0cm}(-1cm,-3.78cm)
  \includegraphics[width=\paperwidth]{#1}
  \end{textblock*}
}
%------------------------------------------------
%--------------SLIDE 2---------------------------
%------------------------------------------------
%-----------------------------------------------------------------------
\begin{frame}
\frametitle{Large Hadron Collider (LHC)}
\begin{itemize}
\item the LHC is a discovery machine
\item the LHC will determine the future course 
of high energy physics
\item to smash protons moving at 99.999999\% 
to the speed of light
\item to create conditions a fraction of a second 
after the big bang
\item High energies allows us:
\begin{itemize}
\item to look deeper into nature (E $\alpha$ 1/size),
("powerful microscopes")
\item to discover new particles with
high(er) mass (E=m$c^2$) 
\item to study the early universe (E=kT)
\item revisit the earlier, hotter, history of our 
universe, searching for a new simplicity ("powerful telescopes")
by observing phenomena and particles no longer observable in 
our everyday experience
\end{itemize}
\end{itemize}
\end{frame}
%------------------------------------------------
%--------------SLIDE 3---------------------------
%------------------------------------------------
\begin{frame}
\frametitle{Compact Muon Solenoid (CMS)}
\begin{columns}
\column{0.6\linewidth}
\includegraphics[width=0.99\linewidth,height=0.99\linewidth]{img/cms_3d_jul10}
\begin{itemize}
\item Multi-purpose detector
\end{itemize}
\column{0.48\linewidth}
\begin{itemize}
\item Distinct features are compact size and superconducting solenoid (3.8 T)
\item Tracker and calorimeters are located inside the solenoid
\item Muon detectors are out side the solenoid
\item All these sub-detectors are arranged in layers around the interaction point
\end{itemize}
\end{columns}
\end{frame}
%----------------------------------------------------
%----------------------------------------------------
%----------------------------------------------------
\begin{frame}
\frametitle{Compact Muon Solenoid (CMS)}
\includegraphics[width=0.99\linewidth]{img/CMS_slice}
%\includegraphics[width=0.45\linewidth,height=0.45\linewidth]{img/CMS_slice}
\begin{itemize}
\item Particle identification: Various types of particles interact with matter differently,
leaving a typical or no signal in specific subdetectors.
\end{itemize}
\end{frame}
%------------------------------------------------
%------------------------------------------------
%------------------------------------------------
\begin{frame}
\frametitle{TOP quark physics}
\begin{itemize}
\item The heaviest quark in the SM, discovered in 1995 at Tevatron.
\item Charge of +2/3, weak isospin of +1/2 and life-time is 5$\times10^{-25}$ sec, so decays before hadronization.
\item The spin information is kept by its decay products and its only quark which provides opportunity to study the properties of a bare quark.
\item Major backgrounds to many important searches beyond the Standard Model (BSM) processes.
\end{itemize}
\includegraphics[width=0.5\linewidth]{img/TopPDF}
\includegraphics[width=0.6\linewidth, height=0.35\linewidth]{img/coffee}
\end{frame}
%------------------------------------------------
%------------------------------------------------
%------------------------------------------------
\begin{frame}
\frametitle{Scope}
\begin{itemize}
\item We aim to measure the cross section at 13 TeV
\begin{itemize}
\item how precise can we measure it in the l+jets final state
\item precise measurement of the cross section opens the possibility to perform
\begin{itemize}
\item precision measurements (PDFs, $\alpha_S$)
\item extrapolating the pole mass \href {http://arxiv.org/abs/1511.00841}{\beamergotobutton{http://arxiv.org/abs/1511.00841}}
\end{itemize}
\end{itemize}
\begin{itemize}
\item search for physics beyond the standard model is particularly relevant at Run 2
\begin{itemize}
\item A deviation in the ratio of cross section at 13 TeV/ 8TeV \href {http://arxiv.org/abs/1206.3557}{\beamergotobutton{http://arxiv.org/abs/1206.3557}}
\item by setting limit on stop production \href {http://arxiv.org/abs/1407.1043}{\beamergotobutton{http://arxiv.org/abs/1407.1043}}
\item A deviation in the branching ratios of different final states (test lepton universality in ttbar decays) 
\item For this preliminary result, we shall measure $\sigma$ only
\end{itemize}
\end{itemize}
\end{itemize}
\end{frame}
%------------------------------------------------
%------------SLIDE 4-----------------------------
%------------------------------------------------
\begin{frame}
\frametitle{L+Jets Final State}
\begin{itemize}
\item The l+jets final state is chosen
\item High statistics, moderate background
\item Branching ratio ($t\bar{t} \rightarrow l\nu q\bar{q} b\bar{b}$) = 30\% + 5.3\% from $tau \rightarrow l\nu\nu$ decays
\item Expect 1 lepton, $\cancel{\it{E}}_{T}$ and 4 jets including at least 2 b-tags
\item We start the analysis with 1 jet (next slide)
\end{itemize}
\begin{center}
 \includegraphics[width=0.4\linewidth]{img/decay}
\end{center}
\end{frame}
%----------------------------------------------------
%------------------------------------------------
%------------SLIDE 5-----------------------------
%------------------------------------------------
\begin{frame}
\frametitle{Analysis Strategy}
\scriptsize
\begin{itemize}
\item We divide the analysis in different categories by counting the jets and b-tags
\item Low jet/btag categories to constrain backgrounds while high jet/btag to fit the signal
\item Analysis is also divided according to lepton charge
\begin{itemize}
\item charge asymmetry is expected for W+Jets
\end{itemize}
\item We don't subdivide the simulation per heavy flavor
\begin{itemize}
\item As we are not having the enough statistics
\end{itemize}
\item similar approach of TOP-11-004/TOP-13-004 ($SH_YFT$, legacy Run-I)
\end{itemize}
\begin{center}
 \includegraphics[width=0.4\linewidth]{img/strategy}
\end{center}
\end{frame}
%------------------------------------------------
%------------SLIDE 6-----------------------------
%------------------------------------------------
\begin{frame}
\frametitle{Physics object selection}
\begin{columns}
\column{0.48\linewidth}
\setbeamercolor{block title}{use=structure,fg=white,bg=purple!75!black}
\setbeamercolor{block body}{use=structure,fg=black,bg=blue!20!white}
\begin{block}{Muon Selection}
\begin{adjustwidth}{-0.5em}{-0.5em}
\tiny
\begin{itemize}
\vspace{.1cm}
\item IsoMu20$||$IsoTkMu20 %(data)\\ IsoMu20\_eta2p1$||$IsoTkMu20\_eta2p1 (MC)
\item Exactly one muon
\begin{itemize}
\tiny
\item $ p_{T}$ \textgreater 30 GeV, $|\eta|$ \textless 2.1, tight ID
\item dB \textless 0.2, dZ \textless 0.5, relIso \textless 0.15
\item SIP3D \textless 4
\end{itemize}
\item Loose Muon Veto
\begin{itemize}
\tiny
\item $ p_{T}$ \textgreater 15 GeV, $|\eta|$ \textless 2.4, relIso \textless 0.25
\end{itemize}
\end{itemize}
\vspace{.2cm}
\end{adjustwidth}
\end{block}
\column{0.48\linewidth}
\setbeamercolor{block title}{use=structure,fg=white,bg=purple!75!black}
\setbeamercolor{block body}{use=structure,fg=black,bg=blue!20!white}
\begin{block}{Electron Selection}
\begin{adjustwidth}{-0.5em}{-0.5em}
  \tiny
  \begin{itemize}
  \vspace{.1cm}
\item Ele22\_eta2p1\_WPLoose\_Gsf %(for data)
%\item Ele27\_eta2p1\_WPLoose\_Gsf (for MC)
\item Exactly one electron
\begin{itemize}
\tiny
\item Using official electron tight ID
\item $ p_{T}$ \textgreater 30 GeV, $|\eta|$ \textless 2.1
\item SIP3D \textless 4
\end{itemize}
\item Using recommended electron veto ID
\end{itemize}
\vspace{.1cm}
\end{adjustwidth}
\end{block}
\end{columns}
\setbeamercolor{block title}{use=structure,fg=white,bg=purple!75!black}
\setbeamercolor{block body}{use=structure,fg=black,bg=blue!20!white}
\begin{block}{Jet Selection}
\vspace{0.2cm}
 \tiny
  \begin{itemize}
\item We require at least one jet
\begin{itemize}
\tiny
\item LOOSE jet ID is applied
\item $ p_{T}$ \textgreater 30 GeV, $|\eta|$ \textless 2.4, Fall15\_25nsV2
\end{itemize}
\item Count number of jets identified as b's
\begin{itemize}
\tiny
\item CSV Discriminator value \textgreater 0.800 (medium working point)
\end{itemize}
\end{itemize}
\vspace{.1cm}
\end{block}
\end{frame}
%------------------------------------------------
%------------SLIDE 7-----------------------------
%------------------------------------------------
\begin{frame}
\frametitle{Data and MC samples}
    \begin{table}[t]
 \vspace{-2.5pt}
      \setlength{\tabcolsep}{1.0pt}
      \tiny
          \begin{tabular}{!{\color{purple}\vrule}l!{\color{purple}\vrule}c!{\color{purple}\vrule}c!{\color{purple}\vrule}}
\arrayrulecolor{purple}\hline
\toprule
Data samples &  Integrated Luminosity\\
\arrayrulecolor{purple}\hline %\midrule
\toprule
/SingleMuon/Run2015C\_25ns-16Dec2015-v1/MINIAOD & {}\\
/SingleElectron/Run2015C\_25ns-16Dec2015-v1/MINIAOD & 2.2 fb\textsuperscript{-1}\\
/SingleMuon/Run2015D-16Dec2015-v1/MINIAOD & {}\\
/SingleElectron/Run2015D-16Dec2015-v1/MINIAOD & {}\\
\toprule
\end{tabular}
\end{table}
    \begin{table}[t]
 \vspace{-8.5pt}
      \setlength{\tabcolsep}{1.0pt}
       \tiny
          \begin{tabular}{!{\color{purple}\vrule}l!{\color{purple}\vrule}c!{\color{purple}\vrule}c!{\color{purple}\vrule}c!{\color{purple}\vrule}}%c!{\color{purple}\vrule}}
          \arrayrulecolor{purple}\hline
           \toprule
MC samples &  Process & Cross section[pb] & Events\\
\toprule
\tiny
/TT\_TuneCUETP8M1\_13TeV-powheg-pythia8 & $t\bar{t}$ & 831.76 & 97994442\\
/TTWJetsToLNu\_TuneCUETP8M1\_13TeV-amcatnloFXFX-madspin-pythia8 & $t\bar{t}$+V & 0.2043 & 250307\\
/TTWJetsToQQ\_TuneCUETP8M1\_13TeV-amcatnloFXFX-madspin-pythia8 & $t\bar{t}$+V & 0.4062 & 831847\\
/TTZToQQ\_TuneCUETP8M1\_13TeV-amcatnlo-pythia8 & $t\bar{t}$+V & 0.5297 & 747000\\
/TTZToLLNuNu\_M-10\_TuneCUETP8M1\_13TeV-amcatnlo-pythia8 & $t\bar{t}$+V & 0.2529 & 394200\\
/ZZ\_TuneCUETP8M1\_13TeV-pythia8 & Multiboson & 16.523 & 985600\\
/WWToLNuQQ\_13TeV-powheg & Multiboson & 49.497 & 6996000\\
/WWTo2L2Nu\_13TeV-powheg & Multiboson & 12.178 & 1979988\\
/WZ\_TuneCUETP8M1\_13TeV-pythia8 & Multiboson & 47.13 & 100000\\
/WJetsToLNu\_TuneCUETP8M1\_13TeV-amcatnloFXFX-pythia8 & W & 61526.7 & 199037280\\
/ST\_tW\_top\_5f\_inclusiveDecays\_13TeV-powheg-pythia8\_TuneCUETP8M1 & tW & 35.6 & 1000000\\
/ST\_tW\_antitop\_5f\_inclusiveDecays\_13TeV-powheg-pythia8\_TuneCUETP8M1 & tW & 35.6 & 999400\\
/ST\_t-channel\_4f\_leptonDecays\_13TeV-amcatnlo-pythia8\_TuneCUETP8M1 & t-ch & 44.33 & 3299200\\
/ST\_t-channel\_antitop\_4f\_leptonDecays\_13TeV-powheg-pythia8\_TuneCUETP8M1 & t-ch & 26.38 & 1630900\\
/DYJetsToLL\_M-50\_TuneCUETP8M1\_13TeV-madgraphMLM-pythia8 & DY & 6025.0 & 247512446\\
/DYJetsToLL\_M-10to50\_TuneCUETP8M1\_13TeV-amcatnloFXFX-pythia8 & DY & 18610 & 76558711\\
/Multijets (data) & & &\\
\toprule
\end{tabular}
\end{table}
\begin{itemize}
\tiny
\item MC samples are normalized by:
$\hat{N}=\mathcal{L}\cdot\sigma\cdot\frac{\sum_{i=1}^{\rm N_{sel}} w_i}{\sum_{i=1}^{\rm N_{gen}} w_i}$
%$\frac{ \sum_{i=1}^{selected} w_i \cdot \sigma \mathcal{L} }{ \sum_{i=1}^{generated} w_i}$
\end{itemize}
\end{frame}
%------------------------------------------------
%--------------SLIDE 8---------------------------
%------------------------------------------------
\begin{frame}
\frametitle{Correction Factors applied to Simulation}
\begin{columns}
\column{0.4\textwidth}
\begin{itemize}
\scriptsize
\item PU reweighting based on 69 minimum bias cross section
\item b-Tag scale factors (76X)
\begin{itemize}
\scriptsize
\item we scale Herwig++ efficiency to match pythia8 and apply the BTV scale factors to that
\end{itemize}
\scriptsize
\item Lepton SF (Official POG), cross checked in this analysis
\item JEC (Fall15\_25nsV2)
\item JES sources \href {https://twiki.cern.ch/twiki/bin/view/CMS/JECUncertaintySources}{\beamergotobutton{uncertainties broken in 29 sources}}
\item JER \href {https://twiki.cern.ch/twiki/bin/view/CMS/JetResolution}{\beamergotobutton{JER twiki}}
%\item JER \href {https://twiki.cern.ch/twiki/bin/view/CMS/JetResolution}{\beamergotobutton{https://twiki.cern.ch/twiki/bin/view/CMS/JetResolution}}
\end{itemize}
\column{0.5\textwidth}
\includegraphics[height=0.67\linewidth]{img/btageff}
%\end{center}
\end{columns}
\end{frame}
%---------------------------------------------------------------
%-----------SLIDE 9------------------------------
%------------------------------------------------
\begin{frame}
\frametitle{Stability of the Selection}
\begin{itemize}
\item The rate of events is very stable as function of run number
\item A couple of runs identified with anomalous rates have very low luminosity ($\sim 1pb^{-1}$)
\end{itemize}
\begin{center}
\includegraphics[width=0.5\linewidth,height=0.3\linewidth]{img/muplus/ratevsrun_4j}
\put(-85,-15){\bf{\small\textcolor{blue}{4J,\bf$\mu^{+}$}}}
%\vspace*{-3cm}
%\hspace*{5.9cm} 
\includegraphics[width=0.5\linewidth,height=0.3\linewidth]{img/eplus/ratevsrun_4j}
\put(-85,-15){\bf{\small\textcolor{blue}{4J, $el^{+}$}}}
\end{center}
\end{frame}
%------------------------------------------------
%------------SLIDE 10----------------------------
%------------------------------------------------
\begin{frame}
\frametitle{Trigger Efficiency Measurements}
\begin{center}
 \includegraphics[width=0.27\linewidth]{img/TnP/mu/SIP_eta}
 \put(-35,35){\bf{\tiny\textcolor{blue}{$\mu$,trigger}}}
 \includegraphics[width=0.27\linewidth]{img/TnP/mu/ISO_SIP_eta}
 \put(-35,35){\bf{\tiny\textcolor{blue}{$\mu$,isolation}}}\\
 \includegraphics[width=0.27\linewidth]{img/TnP/e/Tri_et}
 \put(-35,35){\bf{\tiny\textcolor{blue}{el,trigger}}}
 \includegraphics[width=0.27\linewidth]{img/TnP/e/ISO_et}
 \put(-35,35){\bf{\tiny\textcolor{blue}{el,isolation}}}
\end{center}
%\tiny
\begin{itemize}
%\item SIP3D
\small
\item Using official TnP trees adopted for our selection (Different fitting models tried)
\item Results are used as a cross check of the official SFs
%\item Difference $\approx$ 1\%
\end{itemize}
\end{frame}
%------------------------------------------------
%-----------SLIDE 11------------------------------
%------------------------------------------------
\begin{frame}
\Huge{\textbf {\textcolor {blue}{Background estimation}}}\\
%\Large {\textcolor{blue}{(pre-fit, stats only uncertainty)}}
\end{frame}
%------------------------------------------------
%-----------SLIDE 12------------------------------
%------------------------------------------------
\begin{frame}
\frametitle{Background Estimation: W+Jets}
\begin{itemize}
\item Modelled from simulation
\item Analysis in each lepton channel is sub-divided according to lepton charge
\item $t\bar{t}$ production is expected to be symmetric
\item W production is expected to asymmetric due to PDF composition of protons
\item Assign the corresponding uncertainties from the choice of the QCD scales and PDFs.
\begin{itemize}
\item The templates are currently derived from the MG5\_aMC@NLO
\item Large fraction of negative weights, sufficient statistics to provide positive defined and fairly smooth templates
%\item Try to use shapes from Madgraph instead
\end{itemize}
\item Compared MG5\_aMC@NLO with MADGRAPH in different categories
\item Difference in normalization is observed but the shape of the distributions is fairly similar
\end{itemize}
\end{frame}
%------------------------------------------------
%----------------SLIDE 13------------------------
%------------------------------------------------
\begin{frame}
\frametitle{MET Distributions}
\begin{center}
 \includegraphics[width=0.27\linewidth]{img/wjets/wjets_metpt_1j0t}
 \put(-35,35){\bf{\tiny\textcolor{blue}{1j0t}}}
 \includegraphics[width=0.27\linewidth]{img/wjets/wjets_metpt_2j0t}
 \put(-35,35){\bf{\tiny\textcolor{blue}{2j0t}}}\\
 \includegraphics[width=0.27\linewidth]{img/wjets/wjets_metpt_3j0t}
 \put(-35,35){\bf{\tiny\textcolor{blue}{3j0t}}}
 \includegraphics[width=0.27\linewidth]{img/wjets/wjets_metpt_4j0t}
 \put(-35,35){\bf{\tiny\textcolor{blue}{4j0t}}}
\end{center}
\end{frame}
%------------------------------------------------
%------------------------------------------------
%------------------------------------------------
\begin{frame}
\frametitle{W+Jets Normalization and Charge Asymmetry}
\begin{table}[htb]
\begin{center}
%\label{tab:wjetsnorm}
\begin{tabular}{ lcccc }
\hline
Category & =1 jet & =2 jets & =3 jets & $\geq$4 jets\\
\hline
$\mu^+$ & 1.112$\pm$0.002 & 1.184$\pm$0.006 & 1.20$\pm$0.01 & 0.96$\pm$0.02 \\
$\mu^-$ & 1.095$\pm$0.003 & 1.170$\pm$0.006 & 1.19$\pm$0.01 & 0.97$\pm$0.02 \\
$e^+$ & 1.102$\pm$0.003 & 1.196$\pm$0.007 & 1.21$\pm$0.02 & 0.96$\pm$0.02 \\
$e^-$ & 1.288$\pm$0.004 & 1.38$\pm$0.008 & 1.40$\pm$0.02 & 1.09$\pm$0.03 \\
\hline
\end{tabular}
\end{center}
\end{table}
\begin{itemize}
\item Ratio of the events predicted by MG5\_aMC@NLO wrt MADGRAPH
\item Predicts $\sim$ 20\% more events wrt to madgraph for Njets \textless 4
\item Predicts $\sim$ 5\% less events for Njets \textgreater 4
\item As a consequence of this observation, different charge asymmetry is predicted
\end{itemize}
\end{frame}
%------------------------------------------------
%------------------SLIDE 14----------------------
%------------------------------------------------
\begin{frame}
\frametitle{Background Estimation: QCD Multijets}
\begin{itemize}
\item Normalization is determined from events with low $\cancel{\it{E}}_{T}$ (\textless 20 GeV)
\begin{itemize}
\item Expect to be dominated by QCD
\item Extract QCD shape from events with relIso \textgreater 0.4
\item Normalized from:
\tiny
\begin{equation}
N_{\rm SR}(QCD) = [N_{\rm CR}(obs)-N_{\rm CR}(non-QCD)] \cdot \frac{N_{\rm SR}^{E_T^{miss}<20}(obs)-N_{\rm SR}^{E_T^{miss}<20}(non-QCD)}{N_{\rm CR}^{E_T^{miss}<20}(obs)-N_{\rm CR}^{E_T^{miss}<20}(non-QCD)}\nonumber
%\label{eq:qcdest}
\end{equation}
\end{itemize}
\end{itemize}
\begin{center}
\includegraphics[width=0.4\linewidth]{img/munoniso/metpt_1j}
\put(-55,55){\bf{\tiny\textcolor{blue}{1J, non iso}}}
\includegraphics[width=0.4\linewidth]{img/muplus/metpt_1j}
\put(-55,55){\bf{\tiny\textcolor{blue}{1J, isolated}}}
\end{center}
\end{frame}
%------------------------------------------------
%-----------------SLIDE 15-----------------------
%------------------------------------------------
\begin{frame}
\frametitle{Background Estimation: QCD Multijets}
\begin{table}[htb]
\begin{center}
\begin{tabular}{ lcccc }
\hline
Category &  =1 jet & =2 jets & =3 jets & $\geq$4 jets\\
\hline
$\mu^+$ & $1.89\pm0.09$ & $1.63\pm0.14$ & $2.77\pm0.24$ & $2.17\pm0.23$ \\
$\mu^-$ & $1.59\pm0.05$ & $1.19\pm0.05$ & $2.64\pm0.22$ & $2.01\pm0.15$ \\
$e^+$ & $2.71\pm0.09$ & $2.29\pm0.10$ & $2.60\pm0.13$ & $2.14\pm0.30$ \\
$e^-$ & $2.50\pm0.12$ & $2.06\pm0.05$ & $2.56\pm0.13$ & $2.03\pm0.10$ \\
\hline
\end{tabular}
\end{center}
\end{table}
\begin{itemize}
\item Scale factors to be applied to the control region distributions for QCD to normalize them in the signal region
\item Overall the scale factors are approximately constant across the different jet multiplicity categories
\item tend to increase for higher jet multiplicities
\begin{itemize}
\item Due to NON QCD contamination in control region
%\item Final fit will determine the best normalization for QCD
\end{itemize}
\item Uncertainty is higher in the positively charged lepton sample (as a consequence of charge asymmetry)
\item All scale factors are close to two, data is triggered by isolated lepton triggers
\item We capture only a portion of non-isolated leptons
\end{itemize}
\end{frame}
%------------------------------------------------
%-------------SLIDE 16----------------------------
%------------------------------------------------
\begin{frame}
\Huge{\textbf {\textcolor {blue}{Control distributions}}}\\
\Large {\textcolor{blue}{(pre-fit, stats only uncertainty)}}
\end{frame}
%------------------------------------------------
%-----------SLIDE 17-----------------------------
%------------------------------------------------
\begin{frame}
\frametitle{Events count}
\begin{center}
\begin{tikzpicture}
\node (0,0){\includegraphics[width=1.0\linewidth]{paper_fig/Figure_001}};
\begin{turn}{45}
%\node [opacity=1.9] (0,0) {\scalebox{3.0}{\textcolor{red}{$\displaystyle PAS$}}};
\end{turn}
\end{tikzpicture}
\end{center}
\begin{itemize}
\item e/mu+jets combined for positive/negative charged leptons
\item Statistical uncertainty only
\end{itemize}
\end{frame}
%------------------------------------------------
%-----------SLIDE 18-----------------------------
%------------------------------------------------
\begin{frame}
\frametitle{MET distributions}
\begin{columns}
\column{0.3\textwidth}
\begin{itemize}
\scriptsize
\item At pre-approval (74X), we were using these distributions to constrain backgrounds
\item Not so well modelled in 76X (electron channel)
\begin{itemize}
\scriptsize
\item we count simply the number of events with 0-btags
\end{itemize}
\end{itemize}
\column{0.7\textwidth}
\begin{center}
    \includegraphics[width=0.45\linewidth]{plots/allmetpt_1j0t}
    \put(-65,55){\bf{\tiny\textcolor{blue}{1j0t}}}
    \includegraphics[width=0.45\linewidth]{plots/allmetpt_2j0t}
    \put(-65,55){\bf{\tiny\textcolor{blue}{2j0t}}}\\
    \includegraphics[width=0.45\linewidth]{plots/allmetpt_3j0t}
    \put(-65,55){\bf{\tiny\textcolor{blue}{3j0t}}}
    \includegraphics[width=0.45\linewidth]{plots/allmetpt_4j0t}
    \put(-65,55){\bf{\tiny\textcolor{blue}{4j0t}}}
        \end{center}
\end{columns}
\end{frame}
%------------------------------------------------
%------------------------------------------------
%-----------SLIDE 19-----------------------------
%------------------------------------------------
\begin{frame}
\frametitle{M(l,b)}
\begin{center}
\includegraphics[width=0.33\linewidth]{figs_new/allminmlb_1j1t}
    \put(-35,55){\bf{\tiny\textcolor{blue}{1j1t}}}
\includegraphics[width=0.33\linewidth]{figs_new/allminmlb_2j1t}
    \put(-35,55){\bf{\tiny\textcolor{blue}{2j1t}}}\\
\includegraphics[width=0.33\linewidth]{figs_new/allminmlb_3j1t}
    \put(-35,55){\bf{\tiny\textcolor{blue}{3j1t}}}
\includegraphics[width=0.33\linewidth]{figs_new/allminmlb_4j1t}
    \put(-35,55){\bf{\tiny\textcolor{blue}{4j1t}}}
\end{center}
   \end{frame}
%------------------------------------------------
%--------------SLIDE 20--------------------------
%------------------------------------------------
\begin{frame}
\frametitle{minM(l,b)}
\begin{center}
\includegraphics[width=0.33\linewidth]{figs_new/allminmlb_2j2t}
    \put(-45,55){\bf{\tiny\textcolor{blue}{2j2t}}}
\includegraphics[width=0.33\linewidth]{figs_new/allminmlb_3j2t}
    \put(-45,55){\bf{\tiny\textcolor{blue}{3j2t}}}\\
\includegraphics[width=0.33\linewidth]{figs_new/allminmlb_4j2t}
    \put(-45,55){\bf{\tiny\textcolor{blue}{4j2t}}}
    \end{center}
\end{frame}

%------------------------------------------------
%------------SLIDE 21----------------------------
%------------------------------------------------
%------------------------------------------------
%------------SLIDE 22----------------------------
%------------------------------------------------
%--------------SLIDE 23--------------------------
%------------------------------------------------
\begin{frame}
\Huge{\textbf {\textcolor {blue}{Fit results and cross section measurement}}}
\end{frame}
%------------------------------------------------
%--------------SLIDE 24--------------------------
%------------------------------------------------
\begin{frame}
\frametitle{Cross Section Measurement}
\begin{itemize}
\item The $t\bar{t}$ production cross section $\sigma_{t\bar{t}}$ is extracted from the following expression
\end{itemize}
\hfsetfillcolor{gray!10}
\hfsetbordercolor{gray!50!black}
\begin{equation}\label{e:barwq3}\begin{split}
\sigma=
\tikzmarkin{c}(0.05,0.75)(0.05,0.6)\frac{N_{obs}-N_{bkg}}{A\varepsilon \mathcal{L}}\tikzmarkend{c}\nonumber
\tikz[remember picture] \node[coordinate] (n1) {};
\end{split}\end{equation}

\begin{itemize}
\item We make use of Higgs Combination \tikz[baseline,remember picture]{\node[anchor=base] (t1){tool};}
\item Use all categories to fit the signal and adjust initial background estimation
\item In the fit, uncertainties are included
\begin{equation}
N^{bkg} \rightarrow N_{nom}^{bkg} \cdot (1 +\theta_{btag}) \cdot (1+\theta_{jes}) \cdot ..... \nonumber
\end{equation}
\item Use the profile likelihood method (PLR) to fit the signal strength
\begin{equation}
\mu = \sigma_{obs}/\sigma_{th}\nonumber
\end{equation}
\end{itemize}
\begin{tikzpicture}[remember picture,overlay]   %% use here too
        \path[draw=magenta,thick,->] ([yshift=1mm]n1.west) to [out=0, in=0,distance=0.5in] (t1.east);
\end{tikzpicture}
\end{frame}
%----------------------------------------------------
%---------------SLIDE 25-----------------------------
%----------------------------------------------------
\begin{frame}
\frametitle{Experimental uncertainties}
\begin{minipage}{0.5\textwidth}
\hspace{-1.25cm}
\centering
 \includegraphics[height=0.55\textheight]{img/shapes_4j2t_LtagEff_unc}
    \put(-55,55){\bf{\tiny\textcolor{blue}{$\mu^+$,4j2t}}}
\begin{itemize}
\tiny
\item b-tagging efficiency is expected to have the largest impact on the signal shapes.
\item nuisances are constrained by log-normal distributions
\end{itemize}
 \hspace{-1.25cm}
\end{minipage}%
\begin{minipage}{0.5\textwidth}
\hspace{+0.25cm}
\centering
\includegraphics[height=0.9\textheight]{img/experimental}
\hspace{+0.25cm}
\end{minipage}
\end{frame}
%------------------------------------------------
%------------SLIDE 26----------------------------
%------------------------------------------------
%------------------------------------------------
%------------SLIDE 27----------------------------
%------------------------------------------------
%------------------------------------------------
%------------------------------------------------
%------------------------------------------------
\begin{frame}
\frametitle{Theory uncertainties}
\begin{columns}
\column{0.4\textwidth}
\includegraphics[width=1.1\linewidth]{img/shapes_4j2t_topPt_unc}
\begin{itemize}
\tiny
\item Distributions for the signal shapes
\item Hadronizer is expected to have the largest impact on the signal shapes
\end{itemize}
\column{0.5\textwidth}
%\begin{center}
% \includegraphics[width=0.4\linewidth]{AN}
 \includegraphics[width=1.12\linewidth]{Table_7}
%\end{center}
\end{columns}
\end{frame}
%------------------------------------------------
%------------SLIDE 28----------------------------
%------------------------------------------------
\begin{frame}
\frametitle{Fit result}
\begin{center}
\vspace{-0.35cm}
\includegraphics[scale=.30]{figs_new/nll1dscan_r}
\end{center}
\vspace{-0.5cm}
\begin{itemize}
\small
\item Expected/observed for the cut-in-categories and and the shape analysis
\end{itemize}
\begin{align}
\Aboxed{\mu = 1.003\pm0.003(stat)\pm0.023(syst)}\nonumber
\end{align}
\vspace{-1.0cm}
\begin{align}
\Aboxed{\mu = 1.030\pm0.004(stat)\pm0.034(syst)}\nonumber
\end{align}
\end{frame}
%----------------------------------------------------
%--------------SLIDE 29------------------------------
%----------------------------------------------------
\begin{frame}
\frametitle{post-fit nuisances (expected \& observed)}
\begin{center}
    \includegraphics[width=0.8\linewidth]{plots/nuisances_cinc}
    \put(-85,10){\bf{\tiny\textcolor{blue}{CinC}}}\\
    \includegraphics[width=0.8\linewidth]{plots/nuisances}
    \put(-85,10){\bf{\tiny\textcolor{blue}{shape}}}
    \end{center}
    \begin{itemize}
    \scriptsize
    \item Largest pulls are observed for the main backgrounds W and QCD multijets
    \item All the rest with in 1 $\sigma$
    \item Post fit uncertainties smaller in shapes as expected
    \end{itemize}
\end{frame}
%----------------------------------------------------
%---------------SLIDE 30-----------------------------
%----------------------------------------------------
\begin{frame}
\frametitle{correlation factor between each nuisance and the signal strength}
\begin{center}
    \includegraphics[width=0.8\linewidth]{plots/correlations_1_cinc}
    \put(-85,10){\bf{\tiny\textcolor{blue}{CinC}}}\\
    \includegraphics[width=0.8\linewidth]{plots/correlations_1}
    \put(-85,10){\bf{\tiny\textcolor{blue}{shape}}}
    \end{center}
    \begin{itemize}
    \scriptsize
    \item Largest correlations in main backgrounds W, QCD multijets and luminosity
        \end{itemize}
\end{frame}
%------------------------------------------------
%------------SLIDE 31----------------------------
%------------------------------------------------
\begin{frame}
\frametitle{Fit in different categories}
\begin{center}
\begin{tikzpicture}
\node[anchor=south west,inner sep=0] (image) at (-5,7){\includegraphics[scale=.30]{figs_new/xsecSummary}};
%         \path node at (-1.5,10)[rotate=45,color=red] {\Huge PAS};
 \end{tikzpicture}
\end{center}
\begin{itemize}
\scriptsize
%\item Slight tension (1$\sigma$) between electron and muon channels
%\item Enhanced after combination
\item The QCD contamination makes the W+jets background being pulled by the fit slightly differently in each channel.
\item Difference observed is within uncertainties.
\item Fair agreement between shape and cut-in-categories.
\end{itemize}
\end{frame}
%----------------------------------------------------
%----------------SLIDE 32----------------------------
%----------------------------------------------------
\begin{frame}
\frametitle{Systematics}
%\vspace{-0.02cm}
\center
\tiny
\begin{tabular}{lrr}
\hline
\bf Source & \bf Distributions & \bf Count \\%& \bf Source & \bf Distributions & \bf Count          \\
\hline
Statistical & 0.003 & 0.004 \\
\hline
\multicolumn{3}{c}{\it Experimental uncertainties}\\% & \multicolumn{3}{c}{\it Theoretical uncertainties}\\
Jet energy scale/resolution & 0.001 & 0.007 \\%& $t\bar{t}$ model & 0.002 & 0.001\\
b tagging & 0.005 & 0.011 \\%& Top quark $p_T$ & 0.004 & 0.003\\
Pileup & $<$ 0.001 & $<$ 0.001 \\%& Parton shower scales & 0.001 & 0.005\\
Lepton trigger/selection efficiency & 0.002 & 0.001 \\%& $\mu_{\rm R}/\mu_{\rm F}$ & 0.002 & 0.003\\
Lepton energy scale & 0.001 & $<$ 0.001 \\%& Single top quark model & 0.002 & 0.003\\
W+jets model & 0.019 & 0.022 \\%& Top quark mass &  $<$ 0.001 & 0.001\\
multijet & 0.011 & 0.021 \\
Other backgrounds & 0.001 & 0.001 \\
\hline
\multicolumn{3}{c}{\it Theoretical uncertainties}\\
$t\bar{t}$ model & 0.002 & 0.001\\
Top quark $p_T$ & 0.004 & 0.003\\
Parton shower scales & 0.001 & 0.005\\
$\mu_{\rm R}/\mu_{\rm F}$ & 0.002 & 0.003\\
Single top quark model & 0.002 & 0.003\\
Top quark mass &  $<$ 0.001 & 0.001\\
\hline
\bf Total & \bf 0.023 & \bf 0.034 \\
\hline
\end{tabular}
%\vspace{-0.5cm}
\begin{itemize}
\scriptsize
\item The impacts of the systematic uncertainties on the likelihood are evaluated using the following prescription:
\begin{itemize}
\scriptsize
\item fix the nuisances to their post-fit values
\item repeat the fit freezing several times adding at a time a nuisance (or a group of nuisances)
\end{itemize}
\end{itemize}
\end{frame}
%----------------------------------------------------
%------------------------------------------------
%------------SLIDE 33----------------------------
%------------------------------------------------
\begin{frame}
\frametitle{Cross Section Measurement}
\begin{itemize}
\item The $t\bar{t}$ production cross section $\sigma_{t\bar{t}}$ is extracted from the following expression
\end{itemize}
\hfsetfillcolor{gray!10}
\hfsetbordercolor{gray!50!black}
\begin{equation}\label{e:barwq3}\begin{split}
\sigma=
\frac{N_{obs}-N_{bkg}}{\tikzmarkin{c}(-0.5,-0.25)(-0.03,0.35)A\varepsilon \mathcal{L}}\tikzmarkend{c}\nonumber
\tikz[remember picture] \node[coordinate] (n1) {};
\end{split}\end{equation}

\begin{itemize}
\item Acceptance is estimated from Powheg+Pythia8 $m_t=172.5$ \tikz[baseline,remember picture]{\node[anchor=base] (t1){GeV};}
\begin{itemize}
\item central value affected by the MC choice, hadronizer, parton shower scale
\item PDF uncertainties
\item QCD scale choice at ME level
\end{itemize}
\item Uncertainty on the luminosity is intrinsic to the measurement 2.7 \%
(LUM-15-001)
\end{itemize}
\begin{tikzpicture}[remember picture,overlay]   %% use here too
        \path[draw=magenta,thick,->] ([yshift=1mm]n1.east) to [out=0, in=0,distance=1in] (t1.east);
\end{tikzpicture}
\end{frame}
%------------------------------------------------
%------------------------------------------------
%------------------------------------------------
\begin{frame}
\frametitle{Acceptance Measurement}
\begin{table}[htb]
\tiny
\begin{tabular}{lll}
\hline
 Uncertainty &  Procedure & Impact\\
\hline
PDF+$\alpha_s$  & RMS of the 100 NNPDF3.0 variations with $\Delta \alpha_s$ = $\pm 0.001$ & $\pm 0.0006$\\
QCD scale & Envelope of the variations ($\mu_R,\mu_F$) & $\pm0.0017$\\
PS scale & Difference between nominal PS scale POWHEG+PYTHIA8 and 2,1/2 variations & {}\\
Hadronizer & Difference between PYTHIA8 and HERWIG++ & $\pm0.0039$\\
Integrated luminosity & 2.7\% & {}\\
\hline
\end{tabular}
\end{table}
\vspace{-0.5cm}
\begin{center}
\includegraphics[width=0.4\linewidth]{img/acceptance_powheg}
\end{center}
\vspace{-0.5cm}
%\begin{itemize}
%\tiny
%\item To measure the inclusive cross section, we need to extrapolate from fiducial to full phase space region
%\item Different generator samples studied to check variation of acceptance 
%\item First table indicates the uncertainties considered in the extrapolation to full phase space
%\item We measure the acceptance 0.2972\pm0.0001(stat)\pm0.0043(syst)
%\end{itemize}
\begin{align}
\Aboxed{Acceptance = 0.2972\pm0.0001(stat)\pm0.0043(syst)}\nonumber
\end{align}
\vspace{-0.5cm}
\begin{itemize}
\scriptsize
\item Table indicates the uncertainties considered in the extrapolation to full phase space
\item Dependence on the mass enhanced by the reqquirement of extra jets
\end{itemize}
\end{frame}
%------------------------------------------------
%------------SLIDE 34----------------------------
%------------------------------------------------
\begin{frame}
\frametitle{$t\bar{t}$ cross-section summary}
\begin{columns}
\column{0.4\textwidth}
\tiny
\begin{align}
\Aboxed{\sigma = 834.7 \pm 2.5 (stat) 
\pm 20.7 (syst) 
\pm 22.6 (lumi) 
\pm 12.5 (extrapol)} \nonumber
\end{align}
\vspace{-0.7cm}
\tiny
\begin{align}
\Aboxed{\Delta \sigma /\sigma  = 3.98 \% }\nonumber
\end{align}
%\begin{itemize}
%\small
%\item Thanks J.Kieseler for such a nice plot
%\end{itemize}
\column{0.5\textwidth}
\vspace{-0.17cm}
%\begin{center}
% \includegraphics[width=0.4\linewidth]{AN}
 \includegraphics[width=1.0\linewidth]{img/tt_xsec_cms13_spring16}

%\end{center}
%\vspace{-0.5cm}
\end{columns}
% \begin{align}
%\sigma = 834.7 \pm 2.5 (stat) 
%\pm 20.7 (syst) 
%\pm 22.6 (lumi) 
%\pm 12.5 (extrapol) \nonumber
%\end{align}
\end{frame}
%------------------------------------------------
%------------------------------------------------
%------------------------------------------------
%------------------------------------------------
%------------SLIDE 35----------------------------
%------------------------------------------------
%------------------------------------------------
%------------SLIDE 36----------------------------
%------------------------------------------------
\begin{frame}
\Huge{\textbf {\textcolor {blue}{Pole mass extraction}}}
\end{frame}
%------------------------------------------------
%------------SLIDE 37----------------------------
%------------------------------------------------
\begin{frame}
\frametitle{$top^{++} prediction$}
\begin{center}
    \includegraphics[width=0.5\linewidth]{plots/toppp_param}
%    \put(-85,-10){\bf{\small\textcolor{blue}{CinC}}}
    \end{center}
%    \begin{equation} 
\vspace{-0.25cm}
    \scriptsize
    \begin{align}
\Aboxed{\sigma(m_{t})=\sigma(m_{ref})(\frac{m_{ref}}{m_{t}})^4 \times [1 + a_1 (\frac{m-m_{ref}}{m_{ref}}) + a_2 (\frac{m-m_{ref}}{m_{ref}})^2]}\nonumber
    \end{align}
%\end{equation}
    \begin{itemize}
    \scriptsize
    \item We parameterize the cross section dependency on the pole mass using $TOP^{++}$ v2.0
    \item Thanks J.Kieseler for cross-checking our numbers with $top^{++}$
    \end{itemize}
\end{frame}
%----------------------------------------------------
%-----------------SLIDE 38---------------------------
%----------------------------------------------------
\begin{frame}
\frametitle{Model for the Pole Mass Extraction}
\begin{center}
    \includegraphics[width=0.45\linewidth]{plots/model_polemass}
%    \put(-85,-10){\bf{\small\textcolor{blue}{CinC}}}
    \end{center}
    \begin{itemize}
    \small
    \item The signal strength is re-parametrized as function of the top pole mass
    \begin{itemize}
    \small
    \item model includes an acceptance correction estimated from MC simulation
    \item extra correction to bring the LHCTOPWG reference value to NNPDF 3.0 central value
    \end{itemize}
    \item Thanks to A. David for his help in cross checking our implementation in Higgs combine tool
    \end{itemize}
\end{frame}
%----------------------------------------------------
%----------------------SLIDE 39-----------------------
%-----------------------------------------------------
\begin{frame}
\frametitle{Extra uncertainties in the extraction of pole mass}
\begin{columns}
\column{0.4\textwidth}
\begin{itemize}
\small
\item In addition to the nuisances added to standard fit in the visible phase space we include:
\item 1.5\% extrapolation (lnN)
\item 2.7\% lumi (lnN)
\item QCD scale (lnU)
\item PDF+$\alpha_{s}$ (lnN)
\item Beam energy (lnN)
\end{itemize}
\column{0.5\textwidth}
%\begin{center}
% \includegraphics[width=0.4\linewidth]{AN}
\begin{tikzpicture}
\node[anchor=south west,inner sep=0] (image) at (-5,7){\includegraphics[scale=.35]{Table_2}};
%         \path node at (-1,8)[rotate=45,color=red] {\Huge PAS};
 \end{tikzpicture}
% \includegraphics[width=1.12\linewidth]{Table_2}
%\end{center}
\end{columns}
\end{frame}
%-----------------------------------------------------
%-----------------------------------------------------
%-----------------------------------------------------
%----------------------------------------------------
%-------------------SLIDE 40-------------------------
%----------------------------------------------------
\begin{frame}{Measured top mass}

    \begin{center}

        \begin{tikzpicture}

        \node[anchor=south west,inner sep=0] (image) at (-5,7){\includegraphics[scale=.30]{figs_new/nll1dscan_mpole}};
%         \path node at (-1.5,10)[rotate=45,color=red] {\Huge PAS};


        \end{tikzpicture}

    \end{center}
  \begin{itemize}
    \scriptsize
    \item We measure 172.3 +2.7/-2.3 GeV while we expect an uncertainty +2.3/2.7 GeV
notice the flatness of the likelihood around the minimum. That's induced by a flat prior on the QCD scale uncertainty.
The QCD scale uncertainty is assymetric so it also induces an asymmetry in the uncertainty
    \end{itemize}
\end{frame}
%------------------------------------------------
%------------SLIDE 41-----------------------------
%------------------------------------------------
\begin{frame}
\frametitle{$t\bar{t}$ mass summary}
\begin{columns}
\column{0.4\textwidth}
\begin{itemize}
\item Thanks Benjamin Stieger for this nice summary plot
%\item We measure $\sigma = 834.7 \pm 2.5 (stat) \pm 20.7 (syst) \pm 22.6 (lumi) \pm 12.5 (extrapol)$
%\item $\Delta \sigma /\sigma  = 3.98 \%$
\end{itemize}
\column{0.5\textwidth}
%\begin{center}
% \includegraphics[width=0.4\linewidth]{AN}
 \includegraphics[width=1.1\linewidth]{img/pole_mtop}
%\end{center}
\end{columns}
\end{frame}
%----------------------------------------------------
%-------------------SLIDE 43-------------------------
%----------------------------------------------------
\begin{frame}
\frametitle{Conclusion}
\begin{itemize}
\item We have presented a measurement of the inclusive cross section in the l+jets channel
\begin{itemize}
\item focus on optimizing the data to understand the signal in the visible region
\item uncertainty in the extrapolation reduced by requiring only one jet in the selection
\end{itemize}
\item We measure:
%\setlength{\mathindent}{0pt}
\begin{align}
\Aboxed{\sigma = 834.7 \pm 2.5 (stat) \pm 20.7 (syst) \pm 22.6 (lumi) \pm 12.5 (extrapol)}\nonumber
\end{align}
\vspace{-0.9cm}
\begin{align}
\Aboxed{m_\text{t} = 172.3 +2.7/-2.3 GeV}\nonumber
\end{align}
%$m_t$ = 172.3 +2.7/-2.3 GeV
%\item We request for the approval of analysis for ICHEP
\end{itemize}
\end{frame}
%----------------------------------------------------
%-------------------SLIDE 44-------------------------
%----------------------------------------------------
\begin{frame}
\frametitle{Acknowledgments}
We thank T. Arndt, M. Ivova and A. Anuar for their help with the usage
of the tag and probe tool.
The cross section and modelling and generators sub-conveners
(M. Aldaya, M. Soares, J. Keveaney, E. Yazgan, M. Seidel)  are
acknowledged for their revision of
the current analysis or specific parts of it.
J. Kieseler is thanked for helping cross checking the values obtained
with {TOP++} for the top quark mass dependency of the cross
section.
A. David is thanked for cross checking our implementation of the model
used to parametrize the signal strength as function of the top quark
mass in the Higgs combination tool package.
\end{frame}
%----------------------------------------------------

\begin{frame}
\Huge{\centerline{THE END}}
\end{frame}
%-------------------------------------------------
\end{document} 
